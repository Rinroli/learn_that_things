\documentclass{article}
\usepackage[utf8x]{inputenc}
\usepackage[russian]{babel}
\usepackage{my_style} 
\pagestyle{plain}
\begin{document}
\section{Дифгем}
\begin{definition}[\textbf{Тензорное поле}, №1]
    \emph{Тензорным полем} типа $(p, q)$ и ранга $p+q$ называется объект, который задается в каждой регулярной системе координат $(x)$ набором функций, обозначенных $T_{j_1 \ldots j_q}^{i_1 \ldots i_p}$, со свойством, что при замене координат $(x)\to (x')$ эти функции преобразуются по следующему закону
    $$\underbrace{T_{j'_1 \ldots j'_q}^{i'_1 \ldots i'_p}}_{(x')} = \dd{x^{i'_1}}{x^{i_1}} \cdots \dd{x^{i'_p}}{x^{i_p}} \cdot \dd{x^{j_1}}{x^{j'_1}} \cdots \dd{x^{j_q}}{x^{j'_q}} \cdot \underbrace{T_{j_1 \ldots j_q}^{i_1 \ldots i_p}}_{(x)}$$
    Или в сокращенной записи:
    $$T_{J'}^{I'} = \dd{x^{I'}}{x^I} \cdot \dd{x^J}{x^{J'}} \cdot T_J^I$$
\end{definition}

\begin{theorem}
    Сумма тензорных полей одного типа есть тензорное поле того же типа.
\end{theorem}

\begin{definition}[\textbf{Опускание индекса}, №2]
    Возьмем тензорное поле произвольного вида $T_{j_1 \ldots j_q}^{i_1 \ldots i_p}$
    и невырожденное тензорное поле $g_{ij}(x)$ типа $(0,2)$. Тогда операция
    $$g_{\alpha i_1} \cdot T_{j_1 \ldots j_q}^{i_1 \ldots i_p} = P_{\alpha j_1 \ldots j_q}^{i_2 \ldots i_p}$$
    называется \emph{опусканием индекса}.

    В результате тип тензора преобразуется следующим образом:
    $$(p,q) \to (p-1, q+1)$$
\end{definition}

\begin{definition}[\textbf{Поднятие индекса}, №2]
    Возьмем тензорное поле произвольного вида $T_{j_1 \ldots j_q}^{i_1 \ldots i_p}$ и невырожденное тензорное поле $g_{ij}(x)$ типа $(2,0)$. Тогда операция
    $$g^{\alpha j_1} \cdot T_{j_1 \ldots j_q}^{i_1 \ldots i_p} = Q_{j_2 \ldots j_q}^{\alpha i_1 \ldots i_p}$$
    называется \emph{поднятием индекса}.

    В результате тип тензора преобразуется следующим образом:
    $$(p,q) \to (p+1, q-1)$$
\end{definition}


\end{document}
